\documentclass[stu,12pt,floatsintext]{apa7}

% Language and citation setup
\usepackage[american]{babel}
\usepackage{csquotes}
\usepackage[style=apa,sortcites=true,sorting=nyt,backend=biber]{biblatex}
\DeclareLanguageMapping{american}{american-apa}
\addbibresource{references.bib}

% Font and encoding
\usepackage{fontspec}
\setmainfont{Times New Roman}
\usepackage{unicode-math}
\setmathfont{Latin Modern Math} % or another LuaLaTeX-compatible math font

% PDF tagging for accessibility
\hypersetup{
  pdftitle={Session 5 Discussion Post},
  pdfauthor={Lanie Molinar},
  pdfsubject={Introduction to Systematic Theology (THE-200A)},
  pdfkeywords={Systematic Theology, Discussion Post, APA Style, Session 5},
}

% Document metadata
\title{Session 5 Discussion Post}
\author{Lanie Molinar}
\authorsaffiliations{Colorado Christian University}
\duedate{July 16, 2025}
\course{Introduction to Systematic Theology (THE-200A)}
\professor{Dr. Cari Nimeth}

\begin{document}

\maketitle
\thispagestyle{plain}
\pagestyle{plain}

\subsection{My Personal Concept of Death}

To me, death is a natural part of life, a transition from the physical to the spiritual realm. It is not something to be feared but rather embraced as a necessary step in the cycle of existence. I believe that death allows for the continuation of life in a different form, and it is an opportunity for spiritual growth and transformation. As someone with multiple disabilities and chronic illnesses, I have had to confront the reality of death more than most. This has shaped my understanding of death as a release from physical suffering and a return to a state of peace and wholeness. In fact, as I write this, I am lying in bed, dealing with allergic reactions to a new medication that have left me feeling unwell. I may need to be hospitalized soon, which has prompted me to reflect deeply on my own mortality and the meaning of death in my life. I do not fear death, but rather look forward to the possibility of a new beginning, free from the limitations of my current physical state.

\subsection{How I Am Preparing for Death}

I am preparing for death by working to get closer to God and understanding my faith more deeply. I have been reading the Bible and engaging in prayer, seeking to strengthen my relationship with God. I also find comfort in the idea that death is not the end, but rather a transition to a new existence. I am trying to live my life in a way that reflects my beliefs and values, treating others with kindness and compassion, and striving to leave a positive impact on the world. My mother, who is my caregiver, and I have not discussed death in detail, but I know that she is supportive of my spiritual journey. I have not discussed practical matters such as funeral arrangements or end-of-life care, as my mother is not comfortable with these topics. I myself do not know what I want in terms of a funeral or memorial service, but I trust that my loved ones will honor my wishes when the time comes. I believe that it is important to have these conversations, but I also understand that they can be difficult and uncomfortable. I hope to continue to grow in my faith and understanding of death, so that when the time comes, I can face it with peace and acceptance. Ultimately, I am trusting in God's plan, living by belief and not by sight as we are called to do in \textcite[2 Corinthians 5:7]{Tyndale1996}. In both my understanding of death and my preparation for it, I find peace in the promise of God’s presence and eternal life.

\subsection{References}

\printbibliography

\end{document}